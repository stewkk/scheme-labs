\documentclass[12pt,a4paper,oneside,titlepage]{article}

\usepackage[T2A]{fontenc}
\usepackage[utf8]{inputenc}
\usepackage[russian]{babel}
\usepackage{hyperref}
\usepackage{multicol}
\usepackage{listings}
\usepackage{geometry}

\begin{document}

% НАЧАЛО ТИТУЛЬНОГО ЛИСТА
\begin{center}
  \hfill \break
  \textbf{
    \footnotesize{Министерство науки и высшего образования Российской Федерации}\\
    \hfill \break
    \footnotesize{Федеральное государственное бюджетное образовательное учреждение высшего образования}\\
    \small{«МОСКОВСКИЙ ГОСУДАРСТВЕННЫЙ ТЕХНИЧЕСКИЙ УНИВЕРСИТЕТ имени Н.Э.БАУМАНА\\(национальный исследовательский университет)»}}\\
\end{center}
\hfill \break
\normalsize{Факультет: Информатика и системы управления}\\
\hfill \break
\normalsize{Кафедра: Теоретическая информатика и компьютерные технологии}\\
\hfill\break
\begin{center}
  \textbf{\large{Рубежный контроль №3}}\\
  \large{Конспект по языку \bfseries{Raku}}\\
\end{center}
\hfill \break
\hfill \break
\hfill \break
\begin{flushright}
  \normalsize{
    Выполнил\\
    студент группы ИУ9-11Б\\
    Старовойтов Александр
  }
\end{flushright}
\hfill \break
\hfill \break
\hfill \break
\hfill \break
\hfill \break
\hfill \break
\hfill \break
\hfill \break
\hfill \break
\hfill \break
\hfill \break

\begin{center} Москва, 2021 \end{center}
\thispagestyle{empty} % выключаем отображение номера для этой страницы
% КОНЕЦ ТИТУЛЬНОГО ЛИСТА

\newgeometry{margin=0.5in}
\begin{multicols*}{2}

  \section{Типы}

  Последовательная типизация:

  \begin{itemize}
    \item{\lstinline|my $var;| --- динамическая, неявная}
    \item{\lstinline|my Str $var;| --- статическая, явная}
  \end{itemize}

  \lstinline|$var.WHAT;| --- вернет тип переменной

  Слабая типизация:
  \begin{lstlisting}
    my $var = "2";
    say $var.WHAT; # (Str)
    $var *= 2;
    say $var; # 4
    say $var.WHAT; # (Int)
  \end{lstlisting}

  \section{Управляющие конструкции}

  \begin{itemize}
    \item{\lstinline|if <condition> {...}| --- выполняет блок в \{\}, если <condition>.Bool == True}
    \item{\lstinline|else/elsif| --- альтернативные ветки для if}
    \item{\lstinline|unless <condition> {...}| --- выполняет блок в \{\}, если <condition>.Bool == False}
    \item{\lstinline|with <condition> {...}| --- выполняет блок в \{\}, если <condition> определено}
    \item{\lstinline|for @array -> $item {...}| --- выполняет блок \{\} для каждого элемена @array}
    \item{\begin{lstlisting}
given $var {
    when 0..50 { say '<= 50' }
    default  { say 'default' }
}
\end{lstlisting}
          aналог switch}
    \item{\lstinline|loop (my $i = 0; $i < 5; $i++) {...}| --- for в си-стиле}
  \end{itemize}

  \section{Функциональный стиль}

   Иммутабельность --- префикс \lstinline|constant|:
   \lstinline|my constant $var = 5;|

   Функции --- объекты первого класса:

   \begin{lstlisting}
my $f= -> $n { $n + 3 };
sub g($f, $n) {
    $f($n);
}
say g($f, 7) # 10
   \end{lstlisting}

   Встроенные функции высших порядков для работы с последовательностями:
   \begin{itemize}
     \item{\lstinline|map|
           \begin{lstlisting}
my @array = <1 2 3>;
sub squared($x) {
  $x ** 2
}
say map(&squared,@array); # (1 4 9)
           \end{lstlisting}}
     \item{\lstinline|reduce|
           \begin{lstlisting}
sub f($a, $b) {
    $a + $b;
}
say reduce(&f, (1, 2, 3)); # 6
           \end{lstlisting}}
   \end{itemize}

  \section{IO, строки, регулярные выражения}

  \subsection{IO}
  \lstinline|$*IN, $*OUT, $*ERR| --- хендлеры для stdin, stdout, stderr

  \begin{itemize}
    \item{\lstinline|say|} --- вывести в stdout с переносом строки
    \item{\lstinline|print|} --- вывести в stdout без переноса строки
    \item{\lstinline|note|} --- вывод в stderr
  \end{itemize}

  Пример перенаправления stdout:

  \begin{lstlisting}
my $output-file = open "file.txt", :w;
$*OUT = $output-file;
  \end{lstlisting}

  \subsection{строки}

  \begin{itemize}
    \item{\lstinline|indices|} --- поиск подстроки в строке
    \item{\lstinline|substr|} --- взятие подстроки
    \item{\lstinline|containt|} --- содержит ли подстроку строка
  \end{itemize}

  \subsection{регулярные выражения}

  Оператор \lstinline|~~| - матчит регулярное выражение с строкой:
  \begin{lstlisting}
'enlightenment' ~~ m/ light /
  \end{lstlisting}

\end{multicols*}

\end{document}
